% !TeX root = ../thesis.tex

\chapter{Conclusion and  Outlook}
\label{sec:conclusion_future-work}


\section{Conclusion}

The goal of this thesis was to approach panoptic segmentation as multi-task learning problem whereby learning semantic and instance segmentation jointly and presenting a coherent single view representation unifying semantic segmentation for \textit{stuff} classes and instance segmentation for \textit{things} classes. Presented works explored bottom-up paradigm for panoptic segmentation by conceiving and learning on pixel-level relationships to generate instance representations and subsequently instance segmentation. Since it is desired to replace two-stage object detector based approaches for instance segmentation, presented work studies various instance encoding possibilities in greater detail. A general approach adopted for this purpose is representing instances by their center points as reference along with additional modality that encodes pixel-level distances i.e. offset to center points. To this end, various center points and offset representations have been curated and studied. Research on these representations suggest a favourable size for encoding center point can lead to a better performance. It has also been shown that directly learning the offset representations can be tedious however learning a simple representation while leveraging a brief post-processing step can greatly improve instance boundary delineations and thus better instance segmentation results. Therefore, presented work shows a reasonable performance on panoptic segmentation task whilst avoiding top-down and two-stage instance segmentation approaches. It further highlights the importance for selecting a suitable distribution size for encoding key-points. It also provides a detailed comparison between different encoding styles that can be employed for learning offset vectors. 


\section{Future Work}

It has been observed that center of mass used for selecting the center points for instance representations is robust and generally tackles occlusions to a great extent. However, traffic participants such as bicycles and motorbikes are at times get predicted twice due to larger circular wheels. This is a problem that has been observed in some cases. It is therefore desired to explore the paradigm of center of gravity to represent such center points in future work. 

Since, end to end encoding has still shown relatively better performance quantitatively which to some extent corresponds to a difficult to learn yet more comprehensive representation. Therefore, it makes for an interesting investigation topic to scale such end-to-end offset encodings using an sigmoid function and learn on these scaled offset representations.
